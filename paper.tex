\documentclass{sigchi}

%% EXAMPLE BEGIN -- HOW TO OVERRIDE THE DEFAULT COPYRIGHT STRIP -- (July 22, 2013 - Paul Baumann)
% \toappear{Permission to make digital or hard copies of all or part of this work for personal or classroom use is      granted without fee provided that copies are not made or distributed for profit or commercial advantage and that copies bear this notice and the full citation on the first page. Copyrights for components of this work owned by others than ACM must be honored. Abstracting with credit is permitted. To copy otherwise, or republish, to post on servers or to redistribute to lists, requires prior specific permission and/or a fee. Request permissions from permissions@acm.org. \\
% {\emph{CHI'14}}, April 26--May 1, 2014, Toronto, Canada. \\
% Copyright \copyright~2014 ACM ISBN/14/04...\$15.00. \\
% DOI string from ACM form confirmation}
%% EXAMPLE END -- HOW TO OVERRIDE THE DEFAULT COPYRIGHT STRIP -- (July 22, 2013 - Paul Baumann)

\toappear{}

% Arabic page numbers for submission.  Remove this line to eliminate
% page numbers for the camera ready copy 

%\pagenumbering{arabic}

% Load basic packages
\usepackage{balance}  % to better equalize the last page
\usepackage{graphics} % for EPS, load graphicx instead 
\usepackage[T1]{fontenc}
\usepackage{txfonts}
\usepackage{times}    % comment if you want LaTeX's default font
\usepackage[pdftex]{hyperref}
\usepackage{url}      % llt: nicely formatted URLs
\usepackage{color}
\usepackage{textcomp}
\usepackage{booktabs}
\usepackage{ccicons}
\usepackage{todonotes}
\usepackage{pgf}
\usepackage{tikz}
\usepackage{tkz-graph}
\usepackage{mathtools}

% llt: Define a global style for URLs, rather that the default one
\makeatletter
\def\url@leostyle{%
  \@ifundefined{selectfont}{\def\UrlFont{\sf}}{\def\UrlFont{\small\bf\ttfamily}}}
\makeatother
\urlstyle{leo}

% To make various LaTeX processors do the right thing with page size.
\def\pprw{8.5in}
\def\pprh{11in}
\special{papersize=\pprw,\pprh}
\setlength{\paperwidth}{\pprw}
\setlength{\paperheight}{\pprh}
\setlength{\pdfpagewidth}{\pprw}
\setlength{\pdfpageheight}{\pprh}

% Make sure hyperref comes last of your loaded packages, to give it a
% fighting chance of not being over-written, since its job is to
% redefine many LaTeX commands.
\definecolor{linkColor}{RGB}{6,125,233}
%\hypersetup{%
%  pdftitle={SIGCHI Conference Proceedings Format},
%  pdfauthor={LaTeX},
%  pdfkeywords={SIGCHI, proceedings, archival format},
%  bookmarksnumbered,
%  pdfstartview={FitH},
%  colorlinks,
%  citecolor=black,
%  filecolor=black,
%  linkcolor=black,
%  urlcolor=linkColor,
%  breaklinks=true,
%}

% create a shortcut to typeset table headings
% \newcommand\tabhead[1]{\small\textbf{#1}}

\GraphInit[vstyle=Dijkstra]
\tikzset{
  EdgeStyle/.append style={->,bend left}
}

\hyphenation{PeerMind}

\begin{document}

\title{PeerMind}

%\numberofauthors{3}
%\author{%
%  \alignauthor{1st Author Name\\
%    \affaddr{Affiliation}\\
%    \affaddr{City, Country}\\
%    \email{e-mail address}}\\
%  \alignauthor{2nd Author Name\\
%    \affaddr{Affiliation}\\
%    \affaddr{City, Country}\\
%    \email{e-mail address}}\\
%  \alignauthor{3rd Author Name\\
%    \affaddr{Affiliation}\\
%    \affaddr{City, Country}\\
%    \email{e-mail address}}\\
%}

\maketitle

\begin{abstract}

%\todo[inline]{TODO}

\end{abstract}

%\keywords{Authors' choice; of terms; separated; by semi\-colons;
%  commas, within terms only; this section is required.}

%\category{H.5.m.}{Information Interfaces and Presentation
%  (e.g. HCI)}{Miscellaneous} \category{See
%  \url{http://acm.org/about/class/1998/} for the full list of ACM
%  classifiers. This section is required.}{}{}

\section{Introduction}

The broad idea is that instead of just trying to move the voting process online -- something many others do -- we in
fact add additional value by voting online.
Specifically, we can compute results faster and more accurately, we can use sophisticated statistics to estimate
quantities for the entire voting population, and we can promote clarity and full understanding of the voting items for all.

\todo[inline]{This rest of the section is mostly notes. Organize and improve.}

We can often hear that the issue of our democracy is low engagement.
But this is a fallacy, because we should be instead trying to decrease it.
The best governance system is one where you do not have to be involved yourself, but it is still doing things
like you would like to do them.
% Or even, doing things you might not even know that you would like to be done, especially in a long term.
We want to outsource the job of coordination between people.
Sometimes you want to coordinate people yourself (in your company, team), you might enjoy it, but in most cases
you cannot or do not want to, so you outsource this service, especially at a large scale.

One reason why increasing engagement sounds like a good idea is because current democratic systems have two
questionable assumptions: participants have time to participate, and knowledge to participate.
If you do not participate it is your fault.
Moreover, the legitimacy of decisions is based on participation as well.
This is why those participating in the these democratic systems have motivation to try to increase participation
of others (especially if they believe they will still be in majority), or claim they are at fault to not participate.

engagement is then increased only to assure that people like the results

so if we try to minimize engagement, but still make sure that people like the results, then this is the perfect system

and only in the case when they are on the loosing side, then we have to make sure that they understand the reasons why
they lost (eg. the middle panel in our design) or at least to feel that they participated, in at least some way
(voting, or just delegating)

but the participation is the last on this list of important factors

the more important thing is that they understand why others were thinking differently
or that they do not even have to be engaged

we want that people do not have to pay attention to what government is doing

currently, we are in a loop where we have a useless feedback loop, where government in fact would like for people
to pay some type of attention because they need to be elected, so they have to show like they are doing something,
but on the other hand they do not want too much engagement, because they do not have things in place to really
engage with everyone

government should be like infrastructure, you should not have to care about it while it works, and be there to fix
things if it starts failing, but not even you personally, but have mechanisms and processes in place to find this

Terminology:
\begin{itemize}
\item community -- group which is making a decision using PeerMind
\item member -- user of the app who is participating in decision making
\end{itemize}

\todo[inline]{Terminology: decision vs. vote vs. voting (how to name the step in decision making where you vote, and what is what you cast?)}
\todo[inline]{Terminology: motion? Tally? What is the sum of all votes?}

The main question is: can we increase legitimacy and decrease attention required at the same time.
Many others are trying to increase attention (participation) because they want to increase legitimacy,
assuming that this is the only way.

A list of potential other points:
\begin{itemize}
\item legitimacy (quorum, involvement of more people through delegation)
\item scaling (delegation helps, less attention helps)
\item attention (at same or better decisions)
\item bad assumptions for democracy - time, knowledge, it s rational to know that you do not know, common practice is to ask friends
\item better decisions (more people consulted, attention directed where it should be)
\end{itemize}

Another question: should feedback be visible (to see how others voted before you voted)?
People go with majority.
Better for wisdom of crowds is an independent sample.
Maybe we should only allow how other voted to be seen after you voted (but you can still change the vote),
but we could then observe that and learn from this and see if this is good or bad.
Or we could just store information if you observed existing votes before you voted (you have to click a button to see them).
Maybe going with majority is also good when you want more social cohesion in the community and not so much ``optimal''
wisdom of crowds result -  you would want that people relax their positions based on how other people care.
Cohesion of the group vs. accuracy.
Maybe display only number of votes and confidence, maybe confidence direction or graph, but not average votes.

A list of potential questions:
\begin{itemize}
\item Delegation works?
\item Statistical quorum works?
\item Real-time feedback influences?
\item Private vs. public delegation (do you know how member to whom you delegated is voting)?
\item When delegating, how to assure that you cannot learn how somebody else voted, but that you can still somehow
learn how your vote is used?
\item Knowing how much power you have vs. not knowing -- does it change how people behave? (Knowing might corrupt you, but also motivate you to do more thorough work voting because you know other depend on you.)
\end{itemize}

Performance/theory/scaling in terms in computational (how big could effectively community can be to be able to compute it).
Some other theoretical properties?

Out of scope of this work:

\begin{itemize}
\item How to assure privacy of the delegation network (it is highly sensitive data, probably even more than votes)?
\item How to compute delegation results in a way that it does not require access to the whole network of delegation at once?
\item How can we make this system distributed (and still assure privacy)?
\end{itemize}

\section{Related work}

\cite{andersen2008trust, dirnstorfer2010voting, ford2002delegative, rodriguez2007smartocracy, yamakawa2007toward}

\section{Features}

\subsection{Dynamic quorum}

Using statistics to compute dynamic quorum based on how people vote using hypothesis testing.
If population all votes for one position, then quorum can be much lower than if population is completely split on an
issue, then in fact the whole population should vote because this is the only way we know what the population stands for.
So we can base quorum on the confidentiality.
The goal is to always minimize the amount of information needed from
the population in order to confidently determine what actions the population as a whole wants to take.
This allows faster decision making for non-contentious issues and prevents strategizing with quorum in
referendums, where if there is a quorum a possible best action for you is not to attend.

\todo[inline]{When we have multiple options?}
\todo[inline]{Generalization so that majority and super-majority and consensus is just a parameter.}

\subsection{Score voting}

\todo[inline]{Do we take into the account the distribution of votes? So if you have a very large standard deviation
we might want to prefer a decision less than some other decision with smaller standard deviation, but maybe
also smaller (but still positive) average. Is distribution of votes influencing dynamic quorum?}

\subsection{Delegation}

Because not every member of the community has time or knowledge to vote all the time, each member can delegate
decisions to one or more other members of the community.
When a member does not vote themselves, information how those delegates voted is combined into a vote of
the non-voting member.
If a delegate has not voted themselves either, their vote is computed through delegation as well, recursively.

Each member $M$ can declare their delegates and ratios $r$ in which delegates' votes are combined. For example:

\begin{itemize}
\item Member $M_0$ decides not to delegate at all.
\item Member $M_1$ delegates to $M_0$ and $M_3$ with ratios $0.75$ ($r_{1,0}$) and $0.25$ ($r_{1,3}$), respectively.
\item Member $M_2$ delegates to $M_3$ with ratio $1.00$ ($r_{2,3}$).
\item Member $M_3$ delegates to $M_1$ and $M_2$ with ratios $0.40$ ($r_{3,1}$) and $0.40$ ($r_{3,2}$).
\end{itemize}

Observe that member $M_3$ decided to use only $0.8$ of their vote in the case they do not vote themselves.
The rest ($0.2$) counts towards non-voting population.
In this way a member can decide to participate in voting through delegation, but not fully.
Member $M_0$ decided to not participate at all and if they do not vote they are counted towards non-voting population.

\begin{figure}
  \centering
  \begin{tikzpicture}
    \SetGraphUnit{2.5}

    \Vertex[Math,L=M_0]{M0}
    \EA[Math,L=M_1](M0){M1}
    \SO[Math,L=M_3](M1){M3}
    \WE[Math,L=M_2](M3){M2}

    \Edge[label=$0.40$](M1)(M3)
    \Edge[label=$0.75$](M0)(M1)
    \Edge[label=$0.25$](M3)(M1)

    \tikzset{EdgeStyle/.append style={bend right}}
    \Edge[label=$1.00$](M3)(M2)
    \Edge[label=$0.40$](M2)(M3)
  \end{tikzpicture}
  \caption{Example delegation network. Edges are directed towards a member who is delegating, representing how their
  vote is computed from delegates' votes. E.g., $V_{M_1} = 0.75 V_{M_0} + 0.25 V_{M_3}$}~\label{fig:delegation-network}
\end{figure}

We can represent these delegations as a graph we call \emph{delegation network} as seen in
Figure~\ref{fig:delegation-network}.
We direct edges towards a member who is delegating, to represent how their vote is computed from delegates' votes.
For example, member $M_1$ delegates the decision to members $M_0$ and $M_3$.
$M_0$ and $M_3$ become $M_1$'s delegates.
As a consequence, $M_1$'s vote $V_{M_1}$ is computed from $M_0$'s and $M_3$'s votes ($V_{M_0}$ and $V_{M_3}$).

A delegation network can be described as a system of equations.

\begin{displaymath}
\begin{array}{rcl}
V_{M_0} & = & 0 \\
V_{M_1} & = & r_{1,0} V_{M_0} + r_{1,3} V_{M_3} = 0.75 V_{M_0} + 0.25 V_{M_3} \\
V_{M_2} & = & r_{2,3} V_{M_3} = 1.00 V_{M_3} \\
V_{M_3} & = & r_{3,1} V_{M_1} + r_{3,2} V_{M_2} = 0.40 V_{M_1} + 0.40 V_{M_2} \\
\end{array}
\end{displaymath}

When only some members vote ($\mathbf{v}$) we can use a delegation network to compute votes
of all members ($\mathbf{x}$).
We convert a delegation network to a matrix $\mathrm{D}$:

\begin{displaymath}
\mathrm{D} = \left[d_{ij}\right] = \begin{dcases*}
 1 & if $i = j$ \\
 -r_{i,j} & \parbox[t]{.55\columnwidth}{if $M_i$ did not vote, but delegates to $M_j$ with ratio $r_{i,j}$} \\
 0 & otherwise \\
\end{dcases*}
\end{displaymath}

We solve the matrix equation:

\begin{displaymath}
\mathrm{D} \mathbf{x_v} = \mathbf{v}
\end{displaymath}

To continue with our example, let's say that $M_0$ and $M_2$ voted for a decision with $0.7$ and against with $0.4$,
respectively.
We are using score voting on $[-1, 1]$ range, so those votes are represented with $V_{M_0} = 0.7$ and
$V_{M_2} = -0.4$.

\begin{displaymath}
\mathrm{D} = \left[ \begin{array}{cccc}
1 & 0 & 0 & 0 \\
-0.75 & 1 & -0.25 & 0 \\
0 & 0 & 1 & 0 \\
0 & -0.40 & -0.40 & 1 \\
\end{array} \right]
\end{displaymath}

\begin{displaymath}
\mathbf{v} = \left[ \begin{array}{c}
0.7 \\
0 \\
-0.4 \\
0 \\
\end{array} \right],\quad \mathbf{x_v} = \left[ \begin{array}{c}
0.7 \\
0.54 \\
-0.4 \\
0.056 \\
\end{array} \right]
\end{displaymath}

The result (average of scores) without delegation is $\overline{v} = 0.15$ and
with delegation is $\overline{x} = 0.22$.
Delegation pushes decision slightly in favor.

There are few details we have to look into.
Despite delegation some votes or parts of votes can still happen to be discarded and should count
towards non-voting population.
Members might not declare their delegates, or they might declare them without using their vote fully.
Other members can delegate to them as well, potentially propagating lack of a vote further.
For example, if member $M_0$ would not vote, then their vote would be discarded.
Moreover, we could not use $M_0$'s vote to compute $M_1$'s vote fully, which would influence $M_3$'s vote as well.

One option is that we would normalize ratio in such case, so if $M_0$ does not vote, then $r_{1,3}$ is normalized
from $0.25$ to $1.0$.
We could do this automatically for all members, or we could even leave to members to configure
what to do in such case for them.
In current version of PeerMind we decided to not handle this situation in any special way and we leave
discarded votes to propagate.

\todo[inline]{Or should we be normalizing? In a way we should not normalize if votes are discarded because
member explicitly delegated so ($M_3$ above), but if votes are discarded because we could not compute
all delegations, we might want to automatically normalize.}

As a consequence we have to determine how many votes have been discarded to know the size of the non-voting
population.
Another issue is also a question of abstentions.
In PeerMind members can abstain from voting which is different from simply not participating.
Abstentions are counted towards quorum, but they do not contribute to the result.
To address this, we define two vectors:

\begin{displaymath}
\mathbf{w} = \left[w_{i}\right] = \begin{dcases*}
 1 & if $M_i$ voted \\
 0 & otherwise \\
\end{dcases*},\quad \mathbf{a} = \left[a_{i}\right] = \begin{dcases*}
 1 & if $M_i$ abstained \\
 0 & otherwise \\
\end{dcases*}
\end{displaymath}

Solving $\mathrm{D} \mathbf{x_w} = \mathbf{w}$ and $\mathrm{D} \mathbf{x_a} = \mathbf{a}$ gives
us number of votes $W$ and number of abstentions $A$, respectively, by summing together all values in the
resulting vectors: $W = \sum \mathbf{x_w}$, $A = \sum \mathbf{x_a}$.

\todo[inline]{Is the result then $\frac{\sum \mathbf{x_v}}{W}$ or $\frac{\sum \mathbf{x_v}}{\textrm{number of entries
in}~\mathbf{x_w}~\textrm{which are non-zero}}$?}

\todo[inline]{We also have to explain that with non-zero values of $\mathbf{x_w}$ we also get which zero values in
$\mathbf{x_v}$ are real results and which are zero because delegation could not be computed. One has to use that
to compute delegated votes as well.}

We pass computed votes and number of votes and abstentions into computation of dynamic quorum to determine
a quorum and confidence, in the same manner as if members voted themselves.

% In general, there are various ways to use this delegation scheme.
To whom to delegate could be done for each vote separately, or decisions could be categorized into various
categories and then delegations for that category would be used.
But for simplicity, PeerMind currently supports only defining delegates separately from a decision making process,
and those delegates are then in effect for all votes where a member does not vote themselves.
A member is invited to define initial delegations at a registration time, and can change them later on at will,
but independently from a particular decision being made.

\subsection{Three step process}

\subsection{Real-time feedback}


% Real-time communication in general.

\subsection{Integration with offline meetings}

Decoupling the information presentation and voting processes from the debating process in the full decision-making
pipeline (with the idea being that debating is best done in person, and presenting information/voting is best done online).

\section{Evaluation}

\subsection{Design}

User study/evaluation design.
To create a controlled evaluation of our solutions, to not depend on general usage of our system among communities,
we can evaluate the design with the following user study:
\begin{enumerate}
\item we find a tight community of around 100 people willing to participate in the user study
\item they should know each other well (to know how to delegate)
\item we offer them to pay some Amazon credits or something for participation
\item every member of the community should define delegations
\item we ask them to split into two groups, group A and group B
\item group A (target size is around 20% of the community) will meet together or online and discuss various topics through our system
\item discussed topics some can be completely random, to get people to be familiar with they system, and also have some
easy topics to decide on, like, what ``should be the color of the sky if we could change it'', and some topics which are relevant to the community
\item those discussions would be just for evaluation, but community can use the results in any way they find useful
\item after this discussions and decisions were done we ask everyone from group B to evaluate the results, we present them with two results (blinded, them not knowing which was made in which way)
\begin{itemize}
\item results which would be obtained using traditional methods (majority voting without any our improvements)
\item our results using our system
\end{itemize}
\item we ask them to evaluate which one (or both, or none) of decisions they agree with
\item prediction is that members who have not participated in the decision making process will
find decisions made through our system more often agreeable
\item hypothesis here is that what a decision making system is trying to achieve is to find a decision most of the
whole community can agree with, or like it, even when not everyone participates and not all information is available
on their positions on a topic
\end{enumerate}

\bibliographystyle{sigchi}
\bibliography{refs}

More:
\begin{itemize}
\item https://mason.gmu.edu/~rhanson/futarchy.html
\item http://merkle.com/papers/DAOdemocracyDraft.pdf
\end{itemize}

\end{document}
